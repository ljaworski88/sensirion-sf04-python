\documentclass[12pt]{article}
\usepackage{hyperref}
\usepackage{enumitem}

\newlist{inparaenum}{enumerate}{2} % allow two levels of nesting
\setlist[inparaenum]{nosep} % make the spacing compact
\setlist[inparaenum,1]{label=\bfseries\arabic*.} % use number for top level
\setlist[inparaenum,2]{label=\alph*} % use letter for sublevels

\begin{document}

\tableofcontents

\newpage

\section{Installation Instructions}

\begin{inparaenum}
    \item Download the latest Raspberry Pi image from \url{https://www.raspberrypi.org/downloads/raspbian/}
    \item Mount the image according to the instructions for your system
    \item On the first boot follow the set-up instructions from the Raspberry Pi installer
    \begin{inparaenum}
        \item Set the localization to US
        \item Set a new password. Suggested:\textbf{ortho-lab-169}
        \item Connect to the Wi-Fi and update the system
    \end{inparaenum}
    \item Using a text editor go to the \textbf{boot} folder and open \textit{\textbf{config.txt}}
    \begin{inparaenum}
    \item The full path of the config file is: \textit{\textbf{/boot/config.txt}}
    \end{inparaenum}
    \item add the line: \textbf{dtoverlay=i2c-gpio,bus=3}
    \item Save and close the file
    \item Open the terminal
    \item Run: \textbf{\textit{git clone https://github.com/ljaworski88/sensirion-sf04-python.git}}
    \item Go into the newly created folder and goto the \textbf{src} subfolder
    \begin{inparaenum}
    \item The folder should be in the home directory and is called \textbf{sensirion-sf04-python} 
    \end{inparaenum}
    \item copy \textit{i2c-gpio.dtbo} and move it to \textbf{/boot/overlay} replacing the old version
    %\item Open the terminal and type \textbf{\underline{sudo raspi-config}}.
    %\begin{inparaenum}
        %\item Go to option \textit{\underline{5 Inerfacing Options}}.
        %\item Go to option \textit{\underline{P5 I2C}} and enable it.
    %\end{inparaenum}
    \item Restart the Raspberry Pi
\end{inparaenum}

\end{document}
